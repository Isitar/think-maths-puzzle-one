\documentclass[parskip=half]{scrartcl}
\usepackage[top=2.5cm, bottom=2.5cm, left=2.5cm, right=2.5cm]{geometry}

\usepackage[T1]{fontenc}
\usepackage[utf8]{inputenc}
\usepackage{amsmath, amsfonts}
\usepackage{hyperref}

\title{Think Math Puzzle 1}
\subtitle{Can you spin the table?}
\author{Pascal Lüscher}

\newcommand{\zeroton}{\{0,\ldots,n-1\}}

\let\oldepsilon\epsilon
\renewcommand{\epsilon}{\ensuremath\varepsilon}

\begin{document}
	
	\maketitle
		
	\section{Problem}
	The problem can be found in \href{https://www.think-maths.co.uk/maths-puzzles}{ Matt Parker's Maths Puzzles
}, it's the first one. In his \href{https://www.youtube.com/watch?v=T29dydI97zY}{video} Matt explains the problem very well, so I'm not going to repeat it here. Thanks Matt for the puzzle!
		
	
	\section{Solution}
	While there are always many ways to solve a problem, I opted for an ILP (integer linear program). 
	In this section i present the mathematical formulation and explain it a bit

	\begin{align}
		&\min \quad &&0 \\
		& \text{s.t.} \quad 
		&& \sum_{i=0}^{n-1}x_{i,j}		&&= 1 			&&\quad \forall j \in \zeroton \label{eq:numtakenonce} \\		
		& && \sum_{i=0}^{n-1}x_{j,i}		&&= 1	 		&&\quad \forall j \in \zeroton \label{eq:xonenum} \\
		& && b_{i,j}  					&&= 1 - x_{(j - i \mod n),j}	&&\quad \forall i,j \in \zeroton \label{eq:bassignment} \\
		& && \sum_{i=0}^{n-1}b_{j,i}		&&\geq n-1 		&&\quad \forall j \in \zeroton \label{eq:bsum} \\
		& && x_{0,0}					&&= 1			&& \label{eq:symbreaking}\\
		& && x_{i,j} \in \{0,1\} 		&& 				&&\quad \forall i,j \in \zeroton \label{eq:binary}
	\end{align}
	
	The basic idea is to create a one-hot encoding for the investors($x$) initial-seat (in our case an investor can sit on place $0$ to $n$)
	In the \autoref{eq:numtakenonce} it is ensured that each place is assigned only once per investor. In the \autoref{eq:xonenum} the one-hot encoding is ensured, so investor $1$ can only obtain one seat. 
	The \autoref{eq:bassignment} introduces an expression $b$ which value ranges from $0$ to $1$. It is used to determine if the place matches the investor on the table. For example if the investor 3 sits on place 3 ($x_{3,3} = 1$) the value of $b_{0,3}$ becomes $0$. The first index is a "round" counter, so if the table gets turned once, the index increases. In \autoref{eq:bsum} the sum of all non-matching investors is summed up and it has to be greather than or equal to $n-1$, so only at maximum 1 investor is happy.
	The last constraint, \autoref{eq:symbreaking} is just a symmetry breaking constraint, it strengthens the formulation so the solution $0123456$ is equal to $6012345$ (shifted by $1$).
	
\end{document}